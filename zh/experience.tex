\newcommand{\thuundergrad}{
    \Experience
    {清华大学}
    {主修专业:信息管理与信息系统,经济与管理学院}
    [辅修专业:数据科学与技术,软件学院]
    {2017年9月 - 2021年7月}
    {北京}{
        \item 平均分:3.83/4.0(专业排名:3/29)
        \item 选修课程:会计学原理,经济学原理,公司金融,数据结构与算法,计算机语言与程序设计,程序设计实践等
        \item 获学业优秀奖学金、体育优秀奖学金等多项荣誉
    }
}


\newcommand{\summitview}{
    \Experience
    {武岳峰资本}
    {实习生,投资部}
    {2018年7月 - 2018年8月}
    {上海}{
        \item 独立完成38页的自动驾驶汽车传感器所在行业的研究报告文档,并将重点概括为22页PPT(另附8页补充资料)进行汇报,为团队调研一系列同一行业的投资标的 - 车载毫米波雷达芯片提供支持
        \item 参与了对2个项目CEO的访谈调研,独立完成前期的问题准备,记录并整理访谈内容,综合前期研究结论给出初步的分析意见
        \item 完成各类报告的中英文翻译,根据调查材料汇总制作两份投资项目研究摘要PPT,并完成会议记录
    }
}

\newcommand{\fifaresearch}{
    \Experience
    {虚拟经济中价格指数的构建及货币传导机制研究}
    {清华经管学院,导师:罗文澜教授,刘庆教授}
    {2018年10月 - 至今}
    {北京}{
        \item 与腾讯互联网创新技术联合实验室合作立项,使用Python对FIFA Online 3游戏中,玩家间交易以及玩家与游戏公司交易的逐笔数据进行挖掘和处理,建立理论和计量模型
        \item 分析FIFA Online 3游戏币严重通货膨胀,并造成游戏玩家严重流失以至难以继续运营的原因,为FIFA Online 4的交易平台、货币机制、游戏币奖励等机制设计提出建议
    }
}


\newcommand{\cydp}{
    \Experience
    {CYDP中国青年发展项目}
    {队长}
    {2019年1月 - 2019年2月}
    {纽约}{
        \item 参加了由哥伦比亚大学教授和企业管理者讲演的课程,包括行为经济学、金融风险管理、价值投资、设计思维、市场营销、国际组织等
        \item 拜访了纽约的部分企业,包括摩根士丹利和IBM等
        \item 参与项目创业计划比赛,在队伍中担任队长,统筹任务规划,并使用{\LaTeX} 完成报告的整合与排版,独立运用Hexo搭建\href{https://vopaaz.github.io/STA-Website/}{宣传网站}。最终带领其余8名队员,在7支队伍中取得第二名
    }
}

\newcommand{\tkd}{
    \Experience
    {校跆拳道代表队}
    {队长}
    {2018年5月 - 2019年6月}
    {北京}{
        \item 负责代表队日常训练,假期时联系专业教练进行寒训暑训
        \item 与校文艺部、社团部对接,统筹代表队在校庆等活动中表演节目的策划与排练
        \item 个人最佳成绩:首都高等学校跆拳道精英赛 个人54-kg级 男子竞技第五名
        \item 团队最佳成绩:团体总分亚军、男子团体亚军
    }
}

\newcommand{\semtech}{
    \Experience
    {清华经管学院学生学术发展与科创协会}
    {赛事部部长}
    {2018年5月 - 2019年6月}
    {北京}{
        \item 负责第八届“今经乐道”经济热点分析大赛(校级最高级别赛事)的筹备与策划,主赛场吸引到全国高校200多支队伍参赛,新生赛赛场规模比去年扩大了一倍
        \item 撰写参赛手册中最重要的“大赛主题与选题指导”部分,编写Python脚本高效完成近200份初赛报告的分组、打包、统计评委发回的打分表Excel文件等高度重复性任务
    }
}


\newcommand{\eyds}{
	\Experience
	{安永 NextWave 数据科学挑战赛}
	{队长,地区冠军}
	{2019年4月 - 2019年5月}
	{上海}{
        \item 在全球共有2700余位选手参加的初赛中,使用机器学习技术完成了居民活动路径终点的预测任务。为队伍贡献了80\%的特征工程思路,使用Python独立完成了工作流框架的搭建,并与队友共同完成模型选择和参数调优。最终\href{https://github.com/Vopaaz/EY-DS-Competition}{解决方案}的精准度获得中国地区第四名
        \item 在地区前十名角逐的决赛中,使用Reveal.js制作\href{https://github.com/Vopaaz/EY-DS-Competition-Slides}{演示幻灯片},与队友一同清晰地陈述了整个探究过程的思路,对结果的分析并提出实际应用的可能。依靠最终的展示表现,获得中国地区决赛冠军
    }
}
