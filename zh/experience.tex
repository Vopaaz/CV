\newcommand{\pennmaster}{
    \Experience
    {宾夕法尼亚大学}
    {工学硕士:数据科学,工程与应用科学学院}
    {2021年9月 - 2023年6月}
    {美国 - 费城}{
    \item 选修课程:Statistics for Data Science, Computational Linguistics, Big Data Analytics
    }
}

\newcommand{\thuundergrad}{
    \Experience
    {清华大学}
    {管理学学士:信息管理与信息系统,经济与管理学院}
    [辅修学位:数据科学与技术,软件学院]
    {2017年9月 - 2021年6月}
    {中国 - 北京}{
    \item 平均分:3.84/4.0
    \item 获 2019 年国家奖学金、2018 \& 2020 年学业优秀奖学金等多项荣誉
    }
}

\newcommand{\umnexchange}{
    \Experience
    {明尼苏达大学}
    {交换生:卡尔森管理学院}
    {2019年8月 - 2019年12月}
    {美国 - 明尼阿波里斯}{
    \item 平均分:3.898/4.0
    \item 选修课程:Big Data Engineering and Architecture (研究生课程,在课程大数据分析项目中获得班级第一), Interactive Data Visualization for Business Analytics, Analysis and Modeling of Business Systems
    }
}

\newcommand{\summitview}{
    \Experience
    {武岳峰资本}
    {实习生,投资部}
    {2018年7月 - 2018年8月}
    {中国 - 上海}{
    \item 独立完成 38 页的自动驾驶汽车传感器所在行业的研究报告文档,并将重点概括为 22 页 PPT(另附 8 页补充资料)进行汇报,为团队调研一系列同一行业的投资标的 - 车载毫米波雷达芯片提供支持
    \item 参与了对 2 个项目 CEO 的访谈调研,独立完成前期的问题准备,记录并整理访谈内容,综合前期研究结论给出初步的分析意见
    \item 完成各类报告的中英文翻译,根据调查材料汇总制作两份投资项目研究摘要 PPT,并完成会议记录
    }
}

\newcommand{\continental}{
    \Experience
    {大陆汽车系统(上海)有限公司}
    {实习生,工业工程部}
    {2019年7月 - 2019年8月}
    {中国 - 上海}{
    \item 从 Oracle 数据库中获取数据,使用 MicroStrategy 商业智能工具和 KNIME 数据分析平台对数据进行挖掘。分析并预测一次合格率、物流发货量等重要生产指标,并绘制交互式的可视化图表提供给相关决策者
    \item 深度参与公司机器人流程自动化 (RPA) 战略项目。快速学习并独立使用 UiPath RPA 软件开发了公司试点的三个自动化流程项目。主持了三场内部研讨会,并完成了一份 110 余页的教程文档,推动 RPA 项目在整个公司范围内的实施与应用
    }
}

\newcommand{\bytedance}{
    \Experience
    {字节跳动}
    {大数据研发实习生,1 月 - 6 月}
    [数据科学实习生,6 月 - 8 月]
    {2020年1月 - 2020年8月}
    {中国 - 北京}{
    \item 在公司广告数据平台使用 Hive SQL, Spark 等技术深度参与数据仓库的建模、治理与建设,负责两张数据集市层 Hive 表及其数据链路的维护。主要对接抖音业务线的广告产品 ``Dou+'' 相关的数据需求,支持约 5 个产品新功能相关的数据分析,包括卡券、购物车和直播间推广等
    \item 通过时间序列数据挖掘方法,优化公司 A/B 测试平台的广告指标时序报警机制。将报警准确率从大约 60\% 提升至 90\%
    }
}

\newcommand{\msra}{
    \Experience
    {微软亚洲研究院 (MSRA)}
    {软件分析组,实习研究员}
    {2020年8月 - 2021年3月}
    {中国 - 北京}{
    \item 参与一个云服务异常诊断算法的设计、优化与实现。作为论文第三作者,深度参与文献调研等论文撰写相关工作,并复现其中多篇相关工作进行对比实验。论文《HALO: Hierarchy-aware Fault Localization for Cloud Systems》已被 SIGKDD 2021 接收
    \item 与 Office 365 以及其他内部应用团队合作,基于该算法根据业务需求开发 Flask API 服务或 Kusto (Azure 数据资源管理器) 插件,帮助相关团队将算法整合进各自的报警处理流程。根据反馈,算法有效地缩短了业务团队工程师诊断并处理服务异常的时间。
    }
}

\newcommand{\fifaresearch}{
    \Experience
    {虚拟经济中价格指数的构建及货币传导机制研究}
    {研究导师:罗文澜教授,刘庆教授}
    {2018年10月 - 2019年6月}
    {中国 - 北京}{
    \item 与腾讯互联网创新技术联合实验室合作立项,使用 Python 对 FIFA Online 3 游戏中玩家间交易,以及玩家与游戏公司交易的逐笔数据进行挖掘和处理,建立计量模型
    \item 分析游戏币严重通货膨胀,并造成游戏玩家严重流失以至难以继续运营的原因,为 FIFA Online 4 的交易平台、货币机制、游戏币奖励等机制设计提出建议
    }
}

\newcommand{\umnresearch}{
    \Experience
    {Carlson Leader 时间使用情况分析}
    {研究助理,导师:明尼苏达大学 Prof. Le Zhou}
    {2019年9月 - 2019年12月}
    {美国 - 明尼阿波里斯}{
    \item 使用 Python 提取 500 余份 Excel 文件中的数据,合并为一份 CSV 并进行了数据清洗。帮助完成了描述性统计、相关性和回归等初步数据分析
    \item 使用 Vue.js, Flask 和 MongoDB 搭建了用于替代 Excel 进行实验数据收集的网站,提供用户提交数据以及反馈实验结果的功能。通过用户提交时的验证机制减少了无效数据,同时使界面美观程度显著提升
    }
}

\newcommand{\ryanresearch}{
    \Experience
    {aaa}
    {研究助理,导师:宾夕法尼亚大学沃顿商学院 Prof. Ryan Dew}
    {2021年9月 - 至今}
    {美国 - 宾夕法尼亚}{
    \item aaa
    }
}

\newcommand{\cydp}{
    \Experience
    {CYDP中国青年发展项目}
    {队长}
    {2019年1月 - 2019年2月}
    {美国 - 纽约}{
    \item 参加了由哥伦比亚大学教授和企业管理者讲演的课程,包括行为经济学、金融风险管理、价值投资、设计思维、市场营销、国际组织等
    \item 拜访了纽约的部分企业,包括摩根士丹利和 IBM 等
    \item 参与项目创业计划比赛,在队伍中担任队长,统筹任务规划,并使用 {\LaTeX} 完成报告的整合与排版,独立运用 Hexo 搭建\href{https://vopaaz.github.io/STA-Website/}{宣传网站}。最终带领其余8名队员,在7支队伍中取得第二名
    }
}

\newcommand{\tkd}{
    \Experience
    {校跆拳道代表队}
    {队长}
    {2018年5月 - 2019年6月}
    {中国 - 北京}{
    \item 负责代表队日常训练与招新活动,统筹代表队在校庆等活动中表演节目的策划与排练
    \item 个人最佳成绩:首都高等学校跆拳道精英赛,个人 54-kg 级男子竞技第五名;团队最佳成绩:团体总分亚军
    }
}

\newcommand{\semtech}{
    \Experience
    {清华经管学院学生学术发展与科创协会}
    {赛事部部长}
    {2018年5月 - 2019年6月}
    {中国 - 北京}{
    \item 负责第八届“今经乐道”经济热点分析大赛(校级最高级别赛事)的筹备与策划,主赛场吸引到全国高校 200 多支队伍参赛,新生赛赛场规模比去年扩大了一倍
    \item 撰写参赛手册中最重要的“大赛主题与选题指导”部分,编写 Python 脚本高效完成近200份初赛报告的分组、打包、统计评委发回的打分表 Excel 文件等高度重复性任务
    }
}

\newcommand{\eydatascience}{
    \Experience
    {安永 NextWave 数据科学挑战赛}
    {队长,地区冠军}
    {2019年4月 - 2019年5月}
    {中国 - 上海}{
    \item 带领团队在全球 2700 多名选手中获得中国地区决赛冠军
    \item 使用机器学习技术完成了居民活动路径终点的预测任务。为队伍贡献了 80\% 的特征工程思路,使用 pandas 和 scikit-learn 独立完成了工作流框架的搭建,并与队友共同完成模型选择和参数调优
    \item 组织了一份清晰的展示报告,与队友一同阐述了整个探究过程的思路、对结果进行的分析,并提出实际应用的可能,给评委留下深刻印象并最终帮助我们获得冠军
    }
}

